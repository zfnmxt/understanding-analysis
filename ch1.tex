\documentclass[a4paper]{report}

\usepackage{amsmath}
\usepackage{amsfonts}
\usepackage{amssymb}
\usepackage{amsthm}
\usepackage{enumitem}
\usepackage[sc]{mathpazo}
\linespread{1.05}         % Palladio needs more leading (space between lines)
\usepackage[T1]{fontenc}
\usepackage[margin=1.5in, marginparwidth=2in]{geometry}
\newenvironment{ex}[1]
    {\noindent{\large \bf Exercise #1.}}{\newline}

\begin{document}
\begin{ex}{1.3.2}
\begin{enumerate}[label=\alph*)]
\item Let $A = \{0\}$. Then $\inf A = \sup A = 0$.
\item Impossible. Let the set in question be $\{a_1, a_2, \dots, a_n\}$ and
  suppose $a_s \geq a_i$ for all $i$ (such an $s$ must exist because the set is
  finite). Then $a_s$ is the maximum of the set and hence the set has a supremum.
\item The set $\{\frac{1}{n} \mid n \in \mathbb{Z}\}$ does the trick.
\end{enumerate}
\end{ex}

\begin{ex}{1.3.3}
\begin{enumerate}[label=\alph*)]
\item Since $A$ is bounded below, $B \neq \varnothing$. Furthermore, since $A$
  is nonempty, $B$ must have an upper bound. By the Axiom of Completeness,
  $\sup B$ exists.

  Each $a \in A$ must be an upper bound for $B$ since $b \leq a$. Hence, by part
  ii) of the definition, $\sup B \leq a$ for each $a$, establishing $\sup B$ as
  a lower bound of $A$. Part i) of the definition says that for all lower bounds
  $b \in B$, $b \leq \sup B$. So, we conclude that $\sup B =
  \inf A$.

\item Part a) shows that any nonempty set that is bounded below has an infimum.
\end{enumerate}
\end{ex}

\begin{ex}{1.3.4}
\begin{enumerate}[label=\alph*)]
\item By the Axiom of Completeness, $\sup A_i$ exists for each $i$. 
Let $s_1 = \sup A_2$ and $s_2 = \sup A_2$ and suppose $s_1 \geq s_2$. For each
$a_1 \in A_1$ and $a_2 \in A_2$, $s_1 \geq a_1$ and $s_2 \geq a_2$. Hence, $s_1
\geq a_2$ and we conclude that $s_1$ is an upper bound of $A_1 \cup A_2$.

Let $b$ be an upper bound of $A_1 \cup A_2$. Then, $a_1 \leq b$ and $a_2 \leq
b$. So, $b$ is also an upper bound of $A_1$ and $A_2$. But $s_1 \leq b$ so $\sup
A_1 \cup A_2 = s_1$. 

Similarly, if $s_2 \geq s_1$, we have $ \sup A_1 \cup A_2 = s_2$. We conclude 
\begin{align*}
  \sup(A_1 \cup A_2) = \max(\sup A_1, \sup A_2)
\end{align*}
In general,
\begin{align*}
 \sup\left(\bigcup_{k=1}^n A_k\right) = \max(\sup A_1, \dots, \sup A_n)
\end{align*}
This can be proved by induction on $k$, with the base case ($k = 1$) being
trivial to prove and the inductive step following directly from part a) (where
you have to use the fact that $\max (\max (x_1, \dots, x_n), x_{n+1}) = \max
(x_1, \dots, x_n, x_{n+1})$).
\item No, because an infinite union of bounded sets can result in an unbounded
  set, which has no supremum. For example, 
\begin{align*}
\bigcup_{k=1}^\infty \{ x \mid \abs{x} \leq k\} = \mathbb{R}
\end{align*}
\end{enumerate}
\end{ex}

\begin{ex}{1.3.7}
\item Part i) of the definition is satisfied by assumption. For part ii), let
  $b$ be an upper bound of $A$. Then, for all $a' \in A$, $a' \leq b$ and,
  crucially, $a \leq b$. Hence, $a = \sup A$.
\end{ex}

\begin{ex}{1.3.10}
\begin{enumerate}[label=\alph*)]
\item By the Axiom of Completeness, $c = \sup A$ exists. Furthermore, since $a <
  b$ for all $a \in A, b \in B$, we have that $c \leq b$. But
  $a \leq c$ by definition, so we're done.
\item 
\end{enumerate}
\end{ex}
\end{document}
