\documentclass[a4paper]{report}

\usepackage{amsmath}
\usepackage{amsfonts}
\usepackage{amssymb}
\usepackage{amsthm}
\usepackage{enumitem}
\usepackage[sc]{mathpazo}
\linespread{1.05}         % Palladio needs more leading (space between lines)
\usepackage[T1]{fontenc}
\usepackage[margin=1.5in, marginparwidth=2in]{geometry}
\newenvironment{ex}[1]
    {\noindent{\large \bf Exercise #1.}}{\vspace{0.5cm}}

\begin{document}
\begin{ex}{1.3.2}
\begin{enumerate}[label=\alph*)]
\item Let $A = \{0\}$. Then $\inf A = \sup A = 0$.
\item Impossible. Let the set in question be $\{a_1, a_2, \dots, a_n\}$ and
  suppose $a_s \geq a_i$ for all $i$ (such an $s$ must exist because the set is
  finite). Then $a_s$ is the maximum of the set and hence the set has a supremum.
\item The set $\{\frac{1}{n} \mid n \in \mathbb{Z}\}$ does the trick.
\end{enumerate}
\end{ex}

\begin{ex}{1.3.3}
\begin{enumerate}[label=\alph*)]
\item Since $A$ is bounded below, $B \neq \varnothing$. Furthermore, since $A$
  is nonempty, $B$ must have an upper bound. By the Axiom of Completeness,
  $\sup B$ exists.

  Each $a \in A$ must be an upper bound for $B$ since $b \leq a$. Hence, by part
  ii) of the definition, $\sup B \leq a$ for each $a$, establishing $\sup B$ as
  a lower bound of $A$. Part i) of the definition says that for all lower bounds
  $b \in B$, $b \leq \sup B$. So, we conclude that $\sup B =
  \inf A$.

\item Part a) shows that any nonempty set that is bounded below has an infimum.
\end{enumerate}
\end{ex}

\begin{ex}{1.3.4}
\begin{enumerate}[label=\alph*)]
\item By the Axiom of Completeness, $\sup A_i$ exists for each $i$. 
Let $s_1 = \sup A_2$ and $s_2 = \sup A_2$ and suppose $s_1 \geq s_2$. For each
$a_1 \in A_1$ and $a_2 \in A_2$, $s_1 \geq a_1$ and $s_2 \geq a_2$. Hence, $s_1
\geq a_2$ and we conclude that $s_1$ is an upper bound of $A_1 \cup A_2$.

Let $b$ be an upper bound of $A_1 \cup A_2$. Then, $a_1 \leq b$ and $a_2 \leq
b$. So, $b$ is also an upper bound of $A_1$ and $A_2$. But $s_1 \leq b$ so $\sup
A_1 \cup A_2 = s_1$. 

Similarly, if $s_2 \geq s_1$, we have $ \sup A_1 \cup A_2 = s_2$. We conclude 
\begin{align*}
  \sup(A_1 \cup A_2) = \max(\sup A_1, \sup A_2)
\end{align*}
In general,
\begin{align*}
 \sup\left(\bigcup_{k=1}^n A_k\right) = \max(\sup A_1, \dots, \sup A_n)
\end{align*}
This can be proved by induction on $k$, with the base case ($k = 1$) being
trivial to prove and the inductive step following directly from part a) (where
you have to use the fact that $\max (\max (x_1, \dots, x_n), x_{n+1}) = \max
(x_1, \dots, x_n, x_{n+1})$).
\item No, because an infinite union of bounded sets can result in an unbounded
  set, which has no supremum. For example, 
\begin{align*}
\bigcup_{k=1}^\infty \{ x \mid \abs{x} \leq k\} = \mathbb{R}
\end{align*}
\end{enumerate}
\end{ex}

\begin{ex}{1.3.7}
\item Part i) of the definition is satisfied by assumption. For part ii), let
  $b$ be an upper bound of $A$. Then, for all $a' \in A$, $a' \leq b$ and,
  crucially, $a \leq b$. Hence, $a = \sup A$.
\end{ex}

\begin{ex}{1.3.10}
\begin{enumerate}[label=\alph*)]
\item By the Axiom of Completeness, $c = \sup A$ exists. Furthermore, since $a <
  b$ for all $a \in A, b \in B$, we have that $c \leq b$. But
  $a \leq c$ by definition, so we're done.
\item First, we need to construct disjoint nonempty sets $A, B$ such that $A
  \cup B = \mathbb{R}$. Let $B = \{ x \mid x > e \text{ for all } e \in E\}$.
  Since $E$ is nonempty is bounded above by some $b$, $B$ is nonempty (namely,
  it must contain $b + 1$). Let $A = B^c = {\{ x \mid x \leq e \text{ for all } e
  \in E \}}$. 


  Now, for $a \in A, b \in B, e \in E$, we have $a \leq e$ and $b > e$. Hence,
  $a < b$ as required. By the Cut Property, there exists $c \in \mathbB{R}$ such
  that $a \leq c$ and $c \leq b$. From the definition of $A$, this means that
  for all $e \in E$, $e \leq c$ such that $c$ is an upper bound on $E$, 
  satisfying part i) of the definition of the supremum. Additionally, suppose
  $d$ is an upper bound on $E$. If $d \in A$, then $e \leq d$ and $d
  \notin B$, meaning that $d < b$. This requires that $d = c$: if $c < d$, then
  $a \not\leq c$ for all $a$. Otherwise, $d \in B$, from which it follows that
  $c \leq d.$ We conclude that $c = \sup E$.

\item Let $A = \{ x \in \mathbb{Q} \mid x < \sqrt{2} \}$ and $B = \{x \in
  \mathbb{Q} \mid x > \sqrt{2}\}$. Clearly, $A \cup B = \mathbb{Q}$ (since
  $\sqrt{2} \notin \mathbb{Q}$) and $a < b$
  for all $a \in A$, $b \in B$. By the Cut Property, there exists $c \in
  \mathbb{Q}$ such that for all $a \in A$, $b \in B$, $a \leq c$ and $b \geq c$.
  Since $A \cup B = \mathbb{Q}$ and $A$ and $B$ are disjoint, either $c \in A$
  or $c \in B$. If $c \in A$, then $1 < c < \sqrt{2}$ and
\begin{align*}
c < c + \frac{2 - c^2}{2}  < \sqrt{2} 
\end{align*}
But this means that there's a rational number (namely $c + (2 - c^2)/2$) in $A$ larger
than $c$, which is a contradiction. A similar argument can be made if $c \in B$.
\end{enumerate}
\end{ex}

\begin{ex}{1.3.11}
  \begin{enumerate}[label=\alph*)]
  \item True. Since $A$ and $B$ are nonempty and bounded, $\sup A$ and $\sup B$
    exist. Since $A \subseteq B$, $a \leq \sup B$ for all $a \in A$, i.e., $\sup
    B$ is an upper bound for $A$. Additionally, $\sup A \leq b$ for all
    upper bounds $b$ of $A$. Hence, $\sup A \leq \sup B$.

  \item True. Since $\sup A$ and $\inf B$ exist, $A$ and $B$ must be nonempty
    and bounded. Suppose there does not exist $c \in \mathbb{R}$ such that $a < c <
    b$ for all $a \in A, b \in B$. Then, there exists $a' \in A$ and $b' \in B$
    with $b' \leq a'$. (If $a' < b'$ then $a' < a' + (b'-a')/2 = (a' + b')/2 < b'$ so it must be the case
    that $b' \leq a'$.) But by assumption $a' \leq \sup A < \inf B \leq b'$,
    which is a contradiction.

  \item False. Let $A = B = \varnothing$, then the assumption trivially holds
    but neither $\sup A$ nor $\inf B$ exists.
  \end{enumerate}
\end{ex}

\begin{ex}{1.4.1}
  \begin{enumerate}[label=\alph*)]
    \item Let $a = {p}/{q}$ and $b = {m}/{n}$. Then $ab = (pm)/(qn)$. Since
      $\mathbb{Z}$ is closed under multiplication, $ab \in \mathbb{Q}$. Similar
      for $a + b$, given that $\mathbb{Z}$ is closed under addition.
    \item Suppose $a + t \in \mathbb{Q}$. Then,
      there exists $p, q \in \mathbb{Z}$ such that $a + t = p/q$. Now, since $a
      = m/n$ for some $m,n \mathbb{Z}$, we have $t = p/q - m/n = (np -
      mq)/(qn)$. But this contradicts the irrationality of $t$, so $a + t \in
      \mathbb{I}$. Similar argument for $at \in \mathbb{I}$.
    \item Nothing. By part (b), $1 - \sqrt{2}, \sqrt{2}/2 -1, \sqrt{2}/{2} + 1
      \in \mathbb{I}$. But $(1 - \sqrt(2) + \sqrt(2) = 1 \in \mathbb{Q}$ and
      $(\sqrt{2}/2 -1) + (\sqrt{2}/2 + 1) = \sqrt{2} \in \mathbb{I}$. Similarly,
      $\sqrt{2}\sqrt{2} \in \mathbb{Q}$ but $(\sqrt{2} + 1)\sqrt{2} = 2 +
      \sqrt{2} \in \mathbb{I}$. Hence, $\mathbb{I}$ is neither closed under
      addition nor multiplication.
  \end{enumerate}
\end{ex}

\begin{ex}{1.4.2}
  First, we show that $s$ is an upper bound of $A$. Let $a' \in A$ and suppose
  $a' > s$, which implies $a' - s > 0$. By assumption, $s + 1/n \geq a'$ for all
  $n \in \mathbb{N}$. By the Archimedean Property, there exists $n_0 \in
  \mathbb{N}$ such that $1/n_0 < a' - s$. But this implies $s + 1/n_0 < a'$,
  which is a contradiction. Hence, $s \geq a$ for all $a \in A$.

  Now, we want to show that $s$ is a least upper bound. Let $b$ be an upper
  bound for $A$. By assumption, $s - 1/n < b$ for all $n \in \mathbb{N}$ so $1/n
  > s - b$. Now, suppose $s > b$. Then $s - b > 0$ and so, by the Archimedean
  Property, there exists $n_0 \in \mathbb{N}$ with $1/n_0 < s - b$. But this is
  a contradiction, so $s \leq b$.
\end{ex}

\begin{ex}{1.4.3}
Suppose there exists $x \in \cap_{n=1}^{\infty}(0, 1/n)$. Then, $0 < x < 1/n$
for all $n \in \mathbb{N}$. But, by the Archimedean Property, there exists $n_0
\in \mathbb{N}$ such that $1/n_0 < x$. This is a contradiction, so
$\cap_{n=1}^{\infty}(0, 1/n) = \varnothing$.
\end{ex}

\begin{ex}{1.4.5}
By Theorem 1.4.3, there exists $r \in \mathbb{Q}$ with $a - \sqrt{2} < r < b -
\sqrt{2}$. Hence, $a < r + \sqrt{2} < b$. But, by Exercise 1.4.1 part (a), $r +
\sqrt{2} \in \mathbb{I}$, so we're done.
\end{ex}

\begin{ex}{1.4.7}
  Choose $n_0 \in \mathbb{N}$ large enough so that
  \[ \frac{1}{n_0} < \frac{\alpha^2 - 2}{2\alpha}\]
  Then
  \begin{align*}
  \alpha^2 - \frac{2 \alpha}{n_0} > \alpha^2 - (\alpha^2 - 2) = 2.
  \end{align*}
But this contradicts the fact that $\alpha$ is a least upper bound of $T$.
Hence, we conclude that $\alpha^2 = 2$.
\end{ex}

\begin{ex}{1.5.1}
Let $n_m = \min \left\{n \in \left(\mathbb{N} \setminus \cup_{i=1}^{m-1} n_i \right)
\mid f(n) \in A\right\}$ and let $g(m) = f(n_m)$. Now suppose $g(s) = f(n_s) = f
(n_t) = g(t)$ for some $s, t \in \mathbb{N}$. Because $f$ is 1-1, $n_s = n_t$.
But if $n_s = n_t$ then $s = t$ because if $s < t$ then $n_s \notin
\left(\mathbb{N} \setminus \cup_{i=1}^{t-1} n_i \right)$ and a similar argument
can be made for $t < s$. Hence, $g$ is 1-1. Additionally, $g$ is onto:
if $a \in A$, there exists some $k \in \mathbb{N}$ with $f(k) \in A$ because
$f$ is onto. But there there must be some $m'$ such that $k = n_{m'}$ and hence
$g(m') = f(k) = a$.
\end{ex}

\begin{ex}{1.5.2}
NIP is only true for closed intervals on $\mathbb{R}$ and not $\mathbb{Q}$
(since the proof relies on AoC).
\end{ex}

\begin{ex}{1.5.3}
  \begin{enumerate}[label=\alph*)]
  \item Suppose $A_1$ and $A_2$ are countable and let $B_2 = A_2 \setminus A_1$.
    Since $B_2 \subseteq A_2$, $B_2$ is countable or finite by Theorem 1.5.7.
    Additionally, $A_1 \cup A_2 = A_1 \cup B_2$. Let $f_1 : \mathbb{N}
    \rightarrow A_1$ be 1-1 and onto. If $B_2$ is finite, then $B_2 = \{b_1, b_2,
    \dots, b_s\}$ and we define
    \begin{align*}
      g(n) &= \begin{cases}
        b_n & \text{$n \leq s$} \\
        f_1(n - s) & \text{n > s}
              \end{cases}
    \end{align*}
   $g$ is 1-1: if $g(n) = g(m)$ then either $g(n), g(m) \in B_2$ or $g(n), g(m) \in A_1$
   because $A_1$ and $B_2$ are disjoint. In the former case, we have $b_n = b_m$
   and conclude $n = m$. In the later case, we have $f_1(n - s) = f_1 (m - s)$
   and have $n = m$ since $f$ is 1-1.

   $g$ is onto: If $x \in B_2$ there is some $n$ with $b_n = x$. Hence, $g(n) =
   x$. If $x \in A_1$ then the surjectivity of $f$ gives an $n$ with $f(n) = x$.
   But then $g(n + s) = f_1(n) = x$.

   If $B_2$ is countable with bijection $f_2: \mathbb{N} \rightarrow B_2$, define
   
   \begin{align*}
     g(n) &= \begin{cases}
       f_1(n/2) & \text{$n$ even} \\
       f_2((n - 1)/2) & \text{$n$ odd}
     \end{cases}
   \end{align*}

   $g$ is 1-1: if $g(n) = g(m)$ and $g(n),g(m) \in A_1$ we have $f_1(n/2) = f_2(m/2)$
   and conclude $n = m$. Otherwise, we have $f_2((n - 1)/2) = f_2((m - 1)/2)$
   and have $n = m$ by the injectivity of $f_2$.

   $g$ is onto: Suppose $x \in A_1$. By the surjectivity of $f_1$, we have $n
   \in \mathbb{N}$ with $f_1(n) = x$. Hence, $g(2*n) = f_1(n) = x$. For $x \in
   B_1$, we have some $n$ with $f_2(n) = x$ and have $g(2*n + 1) = f_2(n) = x$
   (since $2*n + 1$ is always odd).

   The more general statement follows by induction on $m$. The inductive step is
   essentially the proof above.
  \item Infinity isn't a number: induction can only be used to prove $\cup_{n=1}^k
    A_n$ is countable for any $k \in \mathbb{N}$.
  \item  Let $R_1$ be the set of integers appearing in the first row of the
    array, $R_2$ in the second, and so on.  Clearly, these sets are all disjoint
    and there are an infinite number of them. Additionally, we'll annotate each
    integer in each $R_i$ with its sequence number via a pairing. For example, $R_1
    = \{ (1, 1), (2, 3), (3, 6), (4, 10), \dots\}$.


    Let $f_i : \mathbb{N} \rightarrow A_i$ be a bijection for each $A_i$. We define our bijective function $g : \mathbb{N} \rightarrow
    \cup_{n=1}^{\inf}A_n$ as
    \begin{align*}
      g(n) = \begin{cases}
        f_1(s) & \text{if $(s, n) \in R_1$ for some $s \in \mathbb{N}$} \\
        f_2(s) & \text{if $(s, n) \in R_2$ for some $s \in \mathbb{N}$} \\
         & \vdots
      \end{cases}
    \end{align*}

  $g$ is 1-1: If $g(n) = g(m)$ then both $g(n),g(m) \in A_i$ for some $i$ since
  the $A_i$'s are disjoint. Hence, $g(n) = f_i(n) = f_i(m) = g(m)$. But each
  $f_i$ is 1-1, so $n = m$.

  $g$ is onto: Let $z \in \cup_{n=1}^{\inf}A_n$. Then there is some $i \in
  \mathbb{N}$ such that $z \in A_i$. But since $f_i$ is onto, there is a $n \in
  \mathbb{N}$ with $f_i(n) = z$. But since $(n, m) \in R_i$ this means that
  $g(m) = f_i(n) = z$. 
\end{enumerate}
\end{ex}
\begin{ex}{1.5.6}
  \begin{enumerate}[label=\alph*)]
    \item The collection $\{(n -1, n) \mid n \in \mathbb{N}\}$ works with
      bijective function $f$ given by $f(n) = (n-1, n)$.
    \item No such collection exists. By Theorem 1.4.3, for any $a, b \in
      \mathbb{R}$ with $a < b$ there is a $r \in \mathbb{Q}$ with $a < r < b$.
      Hence, in any non-empty interval $(a, b)$ we have $r \in (a, b)$. Since
      the collection consists of disjoint intervals, it is easy to construct a
      bijective function from $\mathbb{Q}$ to the collection. But $\mathbb{Q}$
      is countable, so by Exercise 1.5.5, the collection must also be countable.
  \end{enumerate}
\end{ex}

\begin{ex}{1.5.11}
  \begin{enumerate}[label=\alph*)]
  \item Define $h_g : A' \rightarrow B'$ as $g^{-1}$ restricted to $A'$. That
    is, $h_g(a') = b'$ if $g(b') = a'$. Such a $b' \in B'$ exists for all $a'
    \in A$ because $g$ maps $B'$ onto $A'$. $h_g$ is 1-1: if $h_g(a') = h_g(a'')$
    then there are some $b', b'' \in B'$ such that $g(b') = a'$ and $g(b'') =
    a''$.  Hence, $b' = h_g(a') = h_g(a'') = b''$ and we conclude that $a' =
    a''$. $h_g$ is also onto: if $b' \in B'$ then $h_g(g(b')) = b'$. (We're
    guaranteed $g(b') \in A'$ because $g$ maps $B'$ onto $A'$.)

    Define $h_f : A \rightarrow B$ as $f$ restricted to $A$. $h_f$ is onto and
    1-1 by assumption.

    Finally, define
    \begin{align*}
      h(x) &= \begin{cases}
        h_g & \text {if $x \in A'$}, \\
        h_f & \text {if $x \in A$}
              \end{cases}
    \end{align*}
    $h$ is clearly onto and 1-1.
  \item If $A_1 = \varnothing$, then $g$ is onto and we're done (since $Y \sim
    X$ implies $X \sim Y$). So, assume $A_1 \neq \varnothing$. We show $A_n \cap
    A_{n+1} = \varnothing$ for all $n \in \mathbb{N}$ by induction on $n$.

    Base case ($n = 1$): We need to show that $A_1 \cap A_2 = \varnothing$.
    $A_1 = X\setminus g(Y)$ and $A_2 = g(f(A_1)) = g(f(X \setminus g(Y)))$. $A_1$
    consists precisely of all the elements that are \emph{not} in $g$'s
    range--since $A_2 \subseteq g(Y)$ we conclude that $A_1 \cap A_2 =
    \varnothing$.

    Inductive step: Suppose $A_{k} \cap A_{k+1} = \varnothing$ for all $k < n$.
    Note that $f(A_k) \cap f(A_{k+1}) = \varnothing$: if $a_k
    \in A_k$ and $a_{k+1} \in A_{k+1}$ then $f(a_k) \neq f(a_{k+1})$ because $f$
    is 1-1 and $A_{k}, A_{k+1}$ are disjoint. Similarly, $g(f(A_k)) \cap
    g(f(A_{k+1})) = \varnothing$ because $g$ is 1-1. But $A_{k+1} = g(f(A_k))$
    and $A_{k+2} = g(f(A_{k+1}))$, so we're done.

    The fact that the collection $\{f(A_n) \mid n \in \mathbb{N}\}$ follows
    immediately by the fact that $f$ is 1-1.
  \item Let $b \in B = \cup_{n=1}^{\infty} f(A_n)$. Then, there exists $n' \in
    \mathbb{N}$ such that $b \in f(A_{n'})$. But this means there's a $a_{n'}
    \in A_{n'}$ with $f(a_{n'}) = b$.
  \end{enumerate}
\end{ex}
\end{document}
