\documentclass[a4paper]{report}

\usepackage{amsmath}
\usepackage{amsfonts}
\usepackage{amssymb}
\usepackage{amsthm}
\usepackage{enumitem}
\usepackage[sc]{mathpazo}
\linespread{1.05}         % Palladio needs more leading (space between lines)
\usepackage[T1]{fontenc}
\usepackage[margin=1.5in, marginparwidth=2in]{geometry}
\newenvironment{ex}[1]
    {\noindent{\large \bf Exercise #1.}}{\vspace{0.5cm}}

\begin{document}
\begin{ex}{2.2.4}
\begin{enumerate}[label=\alph*)]
\item $(-1, 1, -1, 1, -1, \dot)$
\item Impossibru. An infinite number of ones requires a ``long-term'' behavior
  where the sequence features 1. If the sequence doesn't converge to 1, it also
  has to feature other numbers--but then the sequence is oscillating between 1s
  and these other numbers and hence must be divergent or these other numbers
  must get so close to 1 that the sequence converges to 1.
\item $(1, 2, 2, 3, 3, 3, 4, 4, 4, 4, \dots)$
\end{enumerate}
\end{ex}

\begin{ex}{2.2.5}
  \begin{enumerate}[label=\alph*)]
 \item $\lim a_{n} = 0$. Let $\varepsilon > 0$. Choose $n \in \mathbb{N}$ such
   that $n \geq 5/\varepsilon + 1$. Then,
\begin{align*}
  \left|{a_n - 0}\right| = a_n \leq \left[ \left[ \frac{5}{(5/\varepsilon) + 1} \right] \right] \leq \frac{5}{(5/\varepsilon) + 1}  < \frac{5}{5/\varepsilon} = \varepsilon
\end{align*}
as required.
 \item $\lim a_{n} = 1$. Choose $N \in \mathbb{N}$ such that $N >
   12/(3\varepsilon -1)$ and let $n \geq \mathbb{N}$. Then,
\begin{align*}
  \left| a_n - \frac{4}{3} \right| &<
\left| \left[ \left[ \frac{12 + 4 \left(12/(3 \varepsilon - 1) \right)}{3\left(12/(3 \varepsilon - 1) \right)} \right] \right] - 1 \right| \\
&\leq \frac{12 + 4 \left(12/(3 \varepsilon - 1) \right)}{3\left(12/(3 \varepsilon - 1) \right)} - 1 \\
&= \frac{12 + \left(12/(3 \varepsilon - 1) \right)}{3\left(12/(3 \varepsilon - 1) \right)} \\
&= \frac{3\varepsilon}{3} \\
&= \epsilon
\end{align*}
as required.
  \end{enumerate}
\end{ex}
\begin{ex}{2.2.6}
Suppose $\lim a_n = a$ and also that $\lim a_n = b$ with $a \neq b$. Since $a
\neq b$, there exists $\delta > 0$ such that $| a - b | = \delta$. Now,
by Definition 2.2.3, for every $\epsilon > 0$, it follows that $|a_n - a | <
\epsilon$  and $|a_m - b | < \epsilon$ when $n \geq N$ and $m \geq M$ for some
$M, N$. Choose $\epsilon = \delta/4$ and set $M$ and $N$ appropriately. Now, let
$R = \max\{M, N\}$ and set $r \geq R$. Then, $|a_r - a| < \delta/4$ and $|a_r -
b| < \delta/4$. By the triangle inequality,
\begin{align*}
|a - b| \leq |a_r -a| + |a_r - b| < \delta/2 = \frac{|a - b|}{2}
\end{align*}
which is nonsense because $a - b \neq 0$. By contradiction, $a = b$.
\end{ex}

\begin{ex}{2.2.7}
  \begin{enumerate}[label=\alph*)]
\item Only frequently since $(-1)^n = -1$ for all odd $n$.
\item Eventually implies frequently.
\item A sequence $(a_n)$ converges to $a$ if, given any
  $\varepsilon$-neighborhood $V_{\varepsilon} (a)$ of $a$, $(a_n)$ is eventually
  in $V_\varepsilon (a)$.
\item No, the sequence $(-2)^n$ contains an infinite number of 2s but is not
  eventually in the interval (1.9, 2.1). It is, however, frequently in (1.9,
  2.1). Indeed, any sequence containing an infinite number of 2s must be
  frequently in (1.9, 2.1). If this were not the case, there would be some $N
  \in \mathbb{N}$ such that for all $n \geq N$, $a_n \neq 2$. But then there
  would be at most $N$ 2s in the sequence.
  \end{enumerate}
\end{ex}

\begin{ex}{2.3.1}
\begin{enumerate}[label=\alph*)]
\item By the Algebraic Limit Theorem,
\[0 = \lim(x_n) = \lim(\sqrt{x_n}\sqrt{x_n}) =
  \lim(\sqrt{x})\lim(\sqrt{x})\] so, $\lim(\sqrt{x}) = 0$.
\item Follows by the same argument in part a).
\end{enumerate}
\end{ex}

\begin{ex}{2.3.3}
By the Order Limit Theorem, $\lim y_n \leq \lim z_n = l$. Also, $l = \lim x_n \leq
\lim y_n$. So $l \leq \lim y_n \leq l$ and we conclude $\lim y_n = l$.
\end{ex}

\begin{ex}{2.3.5}
Suppose $(z_n)$ is convergent, i.e. $\lim z_n = z$. Then, for all $\varepsilon >
0$ there's an $N \in \mathbb{N}$ such that for all $n \geq N$, $|z_n - z| <
\varepsilon$. If $n$ is odd, this is the same as $|x_{((n +1) / 2)} - z| <
\varepsilon$. If $n$ is even, this is the same as $|y_{n/2} - z| < \varepsilon$.
Define $n_x = 2n + 1$ and $n_y = 2n$. Clearly, $n_x \geq 2N + 1$ and $n_y \geq
2N$. Additionally, $|x_{n_x} - z| < \varepsilon$ and $|y_{n_y} - z| <
\varepsilon$. We conclude that $\lim x_n = \lim y_n = \lim z_n = z$.

Now suppose that $\lim x_n = \lim y_n = z$.  Let $\varepsilon > 0$. Then, there
exists $N_x, N_y \in \mathbb{N}$ such that for all $n_x \geq N_x$ and $n_y \geq
N_y$, $|x_{n_x} - z| < \varepsilon$ and $|y_{n_y} - z| < \varepsilon$. Set $N =
\max\{2N_x, 2N_y\}$. Choose $n \geq N$. Clearly, $n \geq 2 n_x - 1$ and $n \geq
2 n_y$.  If $n$ is odd, then $|z_n - z| = |x_{(n+1)/2} - z|$. But $(n+1)/2 \geq
n_x$ so $|x_{(n+1)/2} - z| < \varepsilon$. Similarly, if $n$ is even, $|z_n - z|
= |y_{n/2} - z| \varepsilon$. Hence, $|z_n - z| < \varepsilon$ for all $n \geq
N$.
\end{ex}

\begin{ex}{2.3.7}
\begin{enumerate}[label=\alph*)]
\item Let $(x_n) = (n)$ and let $(y_n) = (-n)$. Then, $(x_n + y_n) = (n + (-n))
  = (0)$, which obviously converges.
\item Impossible. By the Algebraic Limit Theorem, $\lim(y_n) = \lim(y_n + x_n -
  x_n) = \lim(y_n + x_n) - \lim(x_n)$.
\item Let $(b_n) = (1/n)$. By Exercise 2.3.6, $\lim (1/n) = 0$.
\item Suppose $(a_n - b_n)$ is bounded. Then, there exists $M > 0$ such that
  $|a_n - b_n| \leq M$ for all $n \in \mathbb{N}$. Similarly, by Theorem 2.3.2,
  there eixsts $B > 0$ such that $|b_n| \leq B$ for all $n \in \mathbb{N}$.
  Since $(a_n)$ is unbounded, for any $K \in \mathbb{R}$, there exists $n_0 \in
  \mathbb{N}$ such that $|a_{n_0}| > K$. So, choose $n_1 \in \mathbb{N}$ such
  that $|a_{n_1}| > M + B$. Then, $|a _{n_1} - b_{n_1} | > M + B - b_{n_1} > M$,
  which is a contradiction.
\item Let $(a_n) = (0)$ and $(b_n) = (-1)^n$. Clearly $(a_nb_n) = (0)$ converges,
  but $(b_n)$ does not.
\end{enumerate}
\end{ex}

\begin{ex}{2.3.8}
  \begin{enumerate}[label=\alph*)]
  \item $p(x) = a_0 + a_1x + a_2x^2 +  \cdots + a_mx^m$. By the
    Algebraic Limit Theorem,
    \begin{align*}
      \lim(p(x_n)) &= \lim(a_0) + \lim(a_1x_n) + \lim(a_2x_n^2) + \cdots + \lim(a_nx_n^m) \\
                   &= \lim(a_0) + \lim(a_1)\lim(x_n) + \lim(a_2)\lim(x_n)^2 + \cdots + \lim(a_m)\lim(x_n)^m \\
                   &= a_0 + a_1x + a_2x^2 + \cdots + a_mx^m \\
                   &= p(x)
    \end{align*}
  \item Let $f(x) = [[x]]$ and $(x_n) = (1.5)$. Clearly, $\lim f(x_n) = 1$ and
    $\lim(x_n) = 1.5$.
  \end{enumerate}
\end{ex}

\begin{ex}{2.3.11}
  \begin{enumerate}[label=\alph*)]
  \item Let $\varepsilon > 0$ and $\lim x_n = x$. We need to find an $N > 0$
    such that for all $n \geq N$,
    \begin{align*}
      |y_n  - x| &= \left| \frac{x_1 + x_2 + \cdots + x_n}{n} - x\right| \\
                 &= \left| \frac{x_1 + x_2 + \cdots + x_n - nx}{n}\right| \\
                 &= \frac{1}{n}\left| (x_1 - x) + (x_2 - x) + \cdots + (x_n - x)\right| \\
                 &\leq \frac{1}{n}\left(|x_1 - x| + |x_2 - x| + \cdots + |x_n - x|\right)  < \varepsilon
    \end{align*}
    Since $(x_n)$ converges, there is an $M > 0$ such that $|x_n - x| <
    \varepsilon/2$ for all $n > M$. Hence, the above becomes
    \begin{align*}
      &\frac{1}{n}\left(|x_1 - x| + |x_2 - x| + \cdots + |x_n - x|\right)  \\
      &= \frac{1}{n}\left( |x_1 - x| + |x_2 - x| + \cdots + |x_{M-1} - x| \right) + \frac{1}{n}\left( |x_{M} - x| + \cdots + |x_n -n | \right)\\
      &< \frac{1}{n}\left( |x_1 - x| + |x_2 - x| + \cdots + |x_{M-1} - x| \right) + \frac{\varepsilon}{2}
    \end{align*}
    Now $(|x_1 -x| + \cdots + |x_{M-1} -x|)$ is finite, so we can choose some $R
    > 0$ large enough such that---with the $1/n$ factor---it's less than $\varepsilon/2$ for all $n \geq
    R$. Namely, 
    \begin{align*}
R = \left[\left[\frac{2(|x_1 -x| + \cdots + |x_{M-1} -x|)}{\varepsilon}\right]\right] + 1
    \end{align*}
We choose $N = \max\{R, M\}$ and then have 
\begin{align*}
|y_n  - x|  &< \frac{1}{n}\left( |x_1 - x| + |x_2 - x| + \cdots + |x_{M-1} - x| \right) + \frac{\varepsilon}{2} \\
            &< \frac{\varepsilon}{2} + \frac{\varepsilon}{2} \\
            &= \varepsilon
\end{align*}
for all $n \geq N$, as required.
\item Consider $(x_n) = (-1)^n$. Then
  \begin{align*}
    y_n &= \frac{(-1) + 1 + (-1) + \cdots + (-1)^n}{n}  = \begin{cases}
      0 & \text{$n$ even} \\
      -1/n & \text{$n$ odd}
       \end{cases}
  \end{align*}
Clearly, $(y_n)$ converges to 0 even though $(x_n)$ does not converge.
  \end{enumerate}
\end{ex}

\begin{ex}{2.3.12}
  \begin{enumerate}[label=\alph*)]
  \item True, follows immediately by part (iii) for the Order Limit Theorem.
  \item True. Suppose $a \in (0, 1)$. Then, $|a_n - a| > 0$ for all $n$ since $a_n \notin
    (0, 1)$. But then we can choose $\varepsilon = \text{argmin}_n(|a _n - a|/2)$ and have
    $|a_n - a| > \epsilon$ for all $n$, contradicting the existence of $a$.
  \item False.  The sequence where the $n$th term consists of the best decimal
    approximation of $\sqrt{2}$ to $n$ places clearly converges to $\sqrt{2}$.
  \end{enumerate}
\end{ex}

\begin{ex}{2.4.1}
  \begin{enumerate}[label=\alph*)]
  \item We'll use induction to show that $x_n \geq x_{n+1}$ for all $n \in
    \mathbb{N}$.

    Base $(n = 1)$: Clearly, $3 \geq 1/(4-3) = 1$.

    Inductive step: Assume $x_n \geq x_{n+1}$ for all $n \leq K$. Then,
    \begin{align*}
      x_K &\leq x_{K+1} \\
      4 - x_K &\geq 4 - x_{K+1} \\
      \frac{1}{4-X_K} &\leq \frac{1}{4 - x_{K+1}} \\
      x_{K+1} &\leq x_{K+2}
    \end{align*}
    as required. Hence, $(x_n)$ is decreasing. Since $3 \geq x_n$ for all $n \in
    \mathbb{N}$, we have that $x_n \geq 0$ and conclude that $|x_n| \leq 3$. By
    Theorem 2.4.2, $(x_n)$ converges.
  \item The sequence $(x_{n+1})$ is just $(x_n)$ without the first
    term--clearly, the converge to the same limit.
  \item  We have,
    \begin{align*}
      \lim x_{n+1} &= \lim \left( \frac{1}{4 - x_n} \right) \\
                   &= \frac{1}{4 - \lim x_n} \tag{by the Algebraic Limit Theorem} \\
                   &= \frac{1}{4 - \lim x_{n+ 1}} \\
    \end{align*}
    Hence, $\lim{x_n} (4 - \lim {x_n}) = 1$ or $\lim {x_n}^2 - 4 \lim{x_n} + 1 =
    0$. The roots are $\frac{4 \pm \sqrt{16 - 4}}{2} = 2 \pm
    \sqrt{3}$. Since $x_{n+1} < x_{n}$ and $x_1 = 3$, we have $\lim x_n = 2 - \sqrt{3}$.
  \end{enumerate}
\end{ex}

\begin{ex}{2.4.2}
  \begin{enumerate}[label=\alph*)]
  \item The argument assumes that the limit exists in the first place (and it
    does not).
  \item Yes, because the limit exists since it is monotone and bounded above (by
    3).
  \end{enumerate}
\end{ex}

\begin{ex}{2.4.3}

  \begin{enumerate}[label=\alph*)]
  \item We have
    \begin{align*}
      x_n &= \sqrt{2 + \sqrt{2 + \sqrt{2 + \sqrt{2 + \cdots \sqrt{2 + \sqrt{2}} }}}} \\
          &<\sqrt{2 + \sqrt{2 + \sqrt{2 + \sqrt{2 + \cdots \sqrt{2 + 2} }}}} \tag{The final $\sqrt{2}$ was replaced with 2} \\
          &= \sqrt{2 + \sqrt{2 + \sqrt{2 + \sqrt{2 + \cdots 2}}}} \\
          &= \sqrt{2 + \sqrt{2 + \sqrt{2 + 2}}} \\ 
          &= 2
    \end{align*}
  Hence, $2$ is an upper bound for the sequence. Additionally, clearly $x_{n+1}
  > x_{n}$ for all $n$, so the sequence is monotone and by Theorem 2.4.2, the
  sequence must converge and $\lim x_n$ exists. Now, $x_{n+1} = \sqrt{2 +
    x_{n}}$, So,
  \begin{align*}
    \lim x_{n+1}^2 &= \lim(2 + x_{n}) \\
    \lim x_{n}^2 &= 2 + \lim x_{n} \tag{By the ALT and since $\lim x_{n+1} = \lim x_n$}\\
    \lim x_n^2 - \lim x_n - 2 = 0 \\
    (\lim x_n +1)(\lim x_n -2) = 0
  \end{align*}
  Hence, $\lim x_n = 2$.
  \item The $n$-th term in the sequence contains $n$ (nested) square roots and
    assume $n > 3$. Then,
    \begin{align*}
      x_n &= \sqrt{2 \sqrt{2 \sqrt{2} \cdots }} = \prod_{k=1}^{n} 2^{1/{2^k}} = 2^{\sum_{k=1}^n 1/2^k}
    \end{align*}
    But $\sum_{k=1}^{\infty} 1/{2^k} = 1$, so that means that the sequence
    $(1/2^k)$ must be bounded, and, correspondingly that $x_n$ must be bounded.
    Hence, $\lim x_n$ must exist. Namely, $\lim x_n = 2$.
  \end{enumerate}
\end{ex}

\begin{ex}{2.4.4}

  \begin{enumerate}[label=\alph*)]
  \item Suppose that $\mathbb{N}$ is bounded from above and consider the
    sequence $(n)$. Clearly, $(n)$ is monotone so by the Monotone Congergence
    Theorem, $\lim n$ exists and we set $\lim n = \alpha$. Now, by the Algebraic
    Limit Theorem, $\lim (n + 1) = \lim n + 1 = \alpha + 1$. But $\lim n = \lim
    (n + 1)$, hence $\alpha = \alpha + 1$, which is a contradiction. We conclude
    that $\mathbb{N}$ is unbounded from above.
  \item In the proof of Theorem 1.4.1, we consider the sequence $(a_n)$, which
    is clearly monotone and bounded above by any $b_n$. Hence the limit $\lim
    a_n = \alpha$ must exist.

    To complete the proof,
    we need to show that for any convergent increasing sequence $(x_n)$, $x_n
    \leq \lim x_n$ for all $n$. Suppose there exists an $n_0$ such that $\lim
    x_n < x_{n_0}$. Since the sequence is increasing, this require that for all
    $n$, $\lim x_n < x_{n + n_0}$. But $\lim x_n = \lim x_{n + n_0}$ and, by
    the Order Limit Theorem, $\lim x_n < \lim x_{n + n_0} = \lim_x$, which is a
    contradiction.

    So, we now have that $a_n \leq \alpha \leq b_n$ for all $n$, as required,
    and the rest of the proof follows just as it did in the AoC version.
  \end{enumerate}
\end{ex}
\end{document}
