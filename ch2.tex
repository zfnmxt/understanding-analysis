\documentclass[a4paper]{report}

\usepackage{amsmath}
\usepackage{amsfonts}
\usepackage{amssymb}
\usepackage{amsthm}
\usepackage{enumitem}
\usepackage[sc]{mathpazo}
\linespread{1.05}         % Palladio needs more leading (space between lines)
\usepackage[T1]{fontenc}
\usepackage[margin=1.5in, marginparwidth=2in]{geometry}
\newenvironment{ex}[1]
    {\noindent{\large \bf Exercise #1.}}{\vspace{0.5cm}}

\begin{document}
\begin{ex}{2.2.4}
\begin{enumerate}[label=\alph*)]
\item $(-1, 1, -1, 1, -1, \dot)$
\item Impossibru. An infinite number of ones requires a ``long-term'' behavior
  where the sequence features 1. If the sequence doesn't converge to 1, it also
  has to feature other numbers--but then the sequence is oscillating between 1s
  and these other numbers and hence must be divergent or these other numbers
  must get so close to 1 that the sequence converges to 1.
\item $(1, 2, 2, 3, 3, 3, 4, 4, 4, 4, \dots)$
\end{enumerate}
\end{ex}

\begin{ex}{2.2.5}
  \begin{enumerate}[label=\alph*)]
 \item $\lim a_{n} = 0$. Let $\varepsilon > 0$. Choose $n \in \mathbb{N}$ such
   that $n \geq 5/\varepsilon + 1$. Then,
\begin{align*}
  \left|{a_n - 0}\right| = a_n \leq \left[ \left[ \frac{5}{(5/\varepsilon) + 1} \right] \right] \leq \frac{5}{(5/\varepsilon) + 1}  < \frac{5}{5/\varepsilon} = \varepsilon
\end{align*}
as required.
 \item $\lim a_{n} = 1$. Choose $N \in \mathbb{N}$ such that $N >
   12/(3\varepsilon -1)$ and let $n \geq \mathbb{N}$. Then,
\begin{align*}
  \left| a_n - \frac{4}{3} \right| &<
\left| \left[ \left[ \frac{12 + 4 \left(12/(3 \varepsilon - 1) \right)}{3\left(12/(3 \varepsilon - 1) \right)} \right] \right] - 1 \right| \\
&\leq \frac{12 + 4 \left(12/(3 \varepsilon - 1) \right)}{3\left(12/(3 \varepsilon - 1) \right)} - 1 \\
&= \frac{12 + \left(12/(3 \varepsilon - 1) \right)}{3\left(12/(3 \varepsilon - 1) \right)} \\
&= \frac{3\varepsilon}{3} \\
&= \varepsilon
\end{align*}
as required.
  \end{enumerate}
\end{ex}
\begin{ex}{2.2.6}
Suppose $\lim a_n = a$ and also that $\lim a_n = b$ with $a \neq b$. Since $a
\neq b$, there exists $\delta > 0$ such that $| a - b | = \delta$. Now,
by Definition 2.2.3, for every $\varepsilon > 0$, it follows that $|a_n - a | <
\varepsilon$  and $|a_m - b | < \varepsilon$ when $n \geq N$ and $m \geq M$ for some
$M, N$. Choose $\varepsilon = \delta/4$ and set $M$ and $N$ appropriately. Now, let
$R = \max\{M, N\}$ and set $r \geq R$. Then, $|a_r - a| < \delta/4$ and $|a_r -
b| < \delta/4$. By the triangle inequality,
\begin{align*}
|a - b| \leq |a_r -a| + |a_r - b| < \delta/2 = \frac{|a - b|}{2}
\end{align*}
which is nonsense because $a - b \neq 0$. By contradiction, $a = b$.
\end{ex}

\begin{ex}{2.2.7}
  \begin{enumerate}[label=\alph*)]
\item Only frequently since $(-1)^n = -1$ for all odd $n$.
\item Eventually implies frequently.
\item A sequence $(a_n)$ converges to $a$ if, given any
  $\varepsilon$-neighborhood $V_{\varepsilon} (a)$ of $a$, $(a_n)$ is eventually
  in $V_\varepsilon (a)$.
\item No, the sequence $(-2)^n$ contains an infinite number of 2s but is not
  eventually in the interval (1.9, 2.1). It is, however, frequently in (1.9,
  2.1). Indeed, any sequence containing an infinite number of 2s must be
  frequently in (1.9, 2.1). If this were not the case, there would be some $N
  \in \mathbb{N}$ such that for all $n \geq N$, $a_n \neq 2$. But then there
  would be at most $N$ 2s in the sequence.
  \end{enumerate}
\end{ex}

\begin{ex}{2.3.1}
\begin{enumerate}[label=\alph*)]
\item By the Algebraic Limit Theorem,
\[0 = \lim(x_n) = \lim(\sqrt{x_n}\sqrt{x_n}) =
  \lim(\sqrt{x})\lim(\sqrt{x})\] so, $\lim(\sqrt{x}) = 0$.
\item Follows by the same argument in part a).
\end{enumerate}
\end{ex}

\begin{ex}{2.3.3}
By the Order Limit Theorem, $\lim y_n \leq \lim z_n = l$. Also, $l = \lim x_n \leq
\lim y_n$. So $l \leq \lim y_n \leq l$ and we conclude $\lim y_n = l$.
\end{ex}

\begin{ex}{2.3.5}
Suppose $(z_n)$ is convergent, i.e. $\lim z_n = z$. Then, for all $\varepsilon >
0$ there's an $N \in \mathbb{N}$ such that for all $n \geq N$, $|z_n - z| <
\varepsilon$. If $n$ is odd, this is the same as $|x_{((n +1) / 2)} - z| <
\varepsilon$. If $n$ is even, this is the same as $|y_{n/2} - z| < \varepsilon$.
Define $n_x = 2n + 1$ and $n_y = 2n$. Clearly, $n_x \geq 2N + 1$ and $n_y \geq
2N$. Additionally, $|x_{n_x} - z| < \varepsilon$ and $|y_{n_y} - z| <
\varepsilon$. We conclude that $\lim x_n = \lim y_n = \lim z_n = z$.

Now suppose that $\lim x_n = \lim y_n = z$.  Let $\varepsilon > 0$. Then, there
exists $N_x, N_y \in \mathbb{N}$ such that for all $n_x \geq N_x$ and $n_y \geq
N_y$, $|x_{n_x} - z| < \varepsilon$ and $|y_{n_y} - z| < \varepsilon$. Set $N =
\max\{2N_x, 2N_y\}$. Choose $n \geq N$. Clearly, $n \geq 2 n_x - 1$ and $n \geq
2 n_y$.  If $n$ is odd, then $|z_n - z| = |x_{(n+1)/2} - z|$. But $(n+1)/2 \geq
n_x$ so $|x_{(n+1)/2} - z| < \varepsilon$. Similarly, if $n$ is even, $|z_n - z|
= |y_{n/2} - z| \varepsilon$. Hence, $|z_n - z| < \varepsilon$ for all $n \geq
N$.
\end{ex}

\begin{ex}{2.3.7}
\begin{enumerate}[label=\alph*)]
\item Let $(x_n) = (n)$ and let $(y_n) = (-n)$. Then, $(x_n + y_n) = (n + (-n))
  = (0)$, which obviously converges.
\item Impossible. By the Algebraic Limit Theorem, $\lim(y_n) = \lim(y_n + x_n -
  x_n) = \lim(y_n + x_n) - \lim(x_n)$.
\item Let $(b_n) = (1/n)$. By Exercise 2.3.6, $\lim (1/n) = 0$.
\item Suppose $(a_n - b_n)$ is bounded. Then, there exists $M > 0$ such that
  $|a_n - b_n| \leq M$ for all $n \in \mathbb{N}$. Similarly, by Theorem 2.3.2,
  there eixsts $B > 0$ such that $|b_n| \leq B$ for all $n \in \mathbb{N}$.
  Since $(a_n)$ is unbounded, for any $K \in \mathbb{R}$, there exists $n_0 \in
  \mathbb{N}$ such that $|a_{n_0}| > K$. So, choose $n_1 \in \mathbb{N}$ such
  that $|a_{n_1}| > M + B$. Then, $|a _{n_1} - b_{n_1} | > M + B - b_{n_1} > M$,
  which is a contradiction.
\item Let $(a_n) = (0)$ and $(b_n) = (-1)^n$. Clearly $(a_nb_n) = (0)$ converges,
  but $(b_n)$ does not.
\end{enumerate}
\end{ex}

\begin{ex}{2.3.8}
  \begin{enumerate}[label=\alph*)]
  \item $p(x) = a_0 + a_1x + a_2x^2 +  \cdots + a_mx^m$. By the
    Algebraic Limit Theorem,
    \begin{align*}
      \lim(p(x_n)) &= \lim(a_0) + \lim(a_1x_n) + \lim(a_2x_n^2) + \cdots + \lim(a_nx_n^m) \\
                   &= \lim(a_0) + \lim(a_1)\lim(x_n) + \lim(a_2)\lim(x_n)^2 + \cdots + \lim(a_m)\lim(x_n)^m \\
                   &= a_0 + a_1x + a_2x^2 + \cdots + a_mx^m \\
                   &= p(x)
    \end{align*}
  \item Let $f(x) = [[x]]$ and $(x_n) = (1.5)$. Clearly, $\lim f(x_n) = 1$ and
    $\lim(x_n) = 1.5$.
  \end{enumerate}
\end{ex}

\begin{ex}{2.3.11}
  \begin{enumerate}[label=\alph*)]
  \item Let $\varepsilon > 0$ and $\lim x_n = x$. We need to find an $N > 0$
    such that for all $n \geq N$,
    \begin{align*}
      |y_n  - x| &= \left| \frac{x_1 + x_2 + \cdots + x_n}{n} - x\right| \\
                 &= \left| \frac{x_1 + x_2 + \cdots + x_n - nx}{n}\right| \\
                 &= \frac{1}{n}\left| (x_1 - x) + (x_2 - x) + \cdots + (x_n - x)\right| \\
                 &\leq \frac{1}{n}\left(|x_1 - x| + |x_2 - x| + \cdots + |x_n - x|\right)  < \varepsilon
    \end{align*}
    Since $(x_n)$ converges, there is an $M > 0$ such that $|x_n - x| <
    \varepsilon/2$ for all $n > M$. Hence, the above becomes
    \begin{align*}
      &\frac{1}{n}\left(|x_1 - x| + |x_2 - x| + \cdots + |x_n - x|\right)  \\
      &= \frac{1}{n}\left( |x_1 - x| + |x_2 - x| + \cdots + |x_{M-1} - x| \right) + \frac{1}{n}\left( |x_{M} - x| + \cdots + |x_n -n | \right)\\
      &< \frac{1}{n}\left( |x_1 - x| + |x_2 - x| + \cdots + |x_{M-1} - x| \right) + \frac{\varepsilon}{2}
    \end{align*}
    Now $(|x_1 -x| + \cdots + |x_{M-1} -x|)$ is finite, so we can choose some $R
    > 0$ large enough such that---with the $1/n$ factor---it's less than $\varepsilon/2$ for all $n \geq
    R$. Namely, 
    \begin{align*}
R = \left[\left[\frac{2(|x_1 -x| + \cdots + |x_{M-1} -x|)}{\varepsilon}\right]\right] + 1
    \end{align*}
We choose $N = \max\{R, M\}$ and then have 
\begin{align*}
|y_n  - x|  &< \frac{1}{n}\left( |x_1 - x| + |x_2 - x| + \cdots + |x_{M-1} - x| \right) + \frac{\varepsilon}{2} \\
            &< \frac{\varepsilon}{2} + \frac{\varepsilon}{2} \\
            &= \varepsilon
\end{align*}
for all $n \geq N$, as required.
\item Consider $(x_n) = (-1)^n$. Then
  \begin{align*}
    y_n &= \frac{(-1) + 1 + (-1) + \cdots + (-1)^n}{n}  = \begin{cases}
      0 & \text{$n$ even} \\
      -1/n & \text{$n$ odd}
       \end{cases}
  \end{align*}
Clearly, $(y_n)$ converges to 0 even though $(x_n)$ does not converge.
  \end{enumerate}
\end{ex}

\begin{ex}{2.3.12}
  \begin{enumerate}[label=\alph*)]
  \item True, follows immediately by part (iii) for the Order Limit Theorem.
  \item True. Suppose $a \in (0, 1)$. Then, $|a_n - a| > 0$ for all $n$ since $a_n \notin
    (0, 1)$. But then we can choose $\varepsilon = \text{argmin}_n(|a _n - a|/2)$ and have
    $|a_n - a| > \varepsilon$ for all $n$, contradicting the existence of $a$.
  \item False.  The sequence where the $n$th term consists of the best decimal
    approximation of $\sqrt{2}$ to $n$ places clearly converges to $\sqrt{2}$.
  \end{enumerate}
\end{ex}

\begin{ex}{2.4.1}
  \begin{enumerate}[label=\alph*)]
  \item We'll use induction to show that $x_n \geq x_{n+1}$ for all $n \in
    \mathbb{N}$.

    Base $(n = 1)$: Clearly, $3 \geq 1/(4-3) = 1$.

    Inductive step: Assume $x_n \geq x_{n+1}$ for all $n \leq K$. Then,
    \begin{align*}
      x_K &\leq x_{K+1} \\
      4 - x_K &\geq 4 - x_{K+1} \\
      \frac{1}{4-X_K} &\leq \frac{1}{4 - x_{K+1}} \\
      x_{K+1} &\leq x_{K+2}
    \end{align*}
    as required. Hence, $(x_n)$ is decreasing. Since $3 \geq x_n$ for all $n \in
    \mathbb{N}$, we have that $x_n \geq 0$ and conclude that $|x_n| \leq 3$. By
    Theorem 2.4.2, $(x_n)$ converges.
  \item The sequence $(x_{n+1})$ is just $(x_n)$ without the first
    term--clearly, the converge to the same limit.
  \item  We have,
    \begin{align*}
      \lim x_{n+1} &= \lim \left( \frac{1}{4 - x_n} \right) \\
                   &= \frac{1}{4 - \lim x_n} \tag{by the Algebraic Limit Theorem} \\
                   &= \frac{1}{4 - \lim x_{n+ 1}} \\
    \end{align*}
    Hence, $\lim{x_n} (4 - \lim {x_n}) = 1$ or $\lim {x_n}^2 - 4 \lim{x_n} + 1 =
    0$. The roots are $\frac{4 \pm \sqrt{16 - 4}}{2} = 2 \pm
    \sqrt{3}$. Since $x_{n+1} < x_{n}$ and $x_1 = 3$, we have $\lim x_n = 2 - \sqrt{3}$.
  \end{enumerate}
\end{ex}

\begin{ex}{2.4.2}
  \begin{enumerate}[label=\alph*)]
  \item The argument assumes that the limit exists in the first place (and it
    does not).
  \item Yes, because the limit exists since it is monotone and bounded above (by
    3).
  \end{enumerate}
\end{ex}

\begin{ex}{2.4.3}

  \begin{enumerate}[label=\alph*)]
  \item We have
    \begin{align*}
      x_n &= \sqrt{2 + \sqrt{2 + \sqrt{2 + \sqrt{2 + \cdots \sqrt{2 + \sqrt{2}} }}}} \\
          &<\sqrt{2 + \sqrt{2 + \sqrt{2 + \sqrt{2 + \cdots \sqrt{2 + 2} }}}} \tag{The final $\sqrt{2}$ was replaced with 2} \\
          &= \sqrt{2 + \sqrt{2 + \sqrt{2 + \sqrt{2 + \cdots 2}}}} \\
          &= \sqrt{2 + \sqrt{2 + \sqrt{2 + 2}}} \\ 
          &= 2
    \end{align*}
  Hence, $2$ is an upper bound for the sequence. Additionally, clearly $x_{n+1}
  > x_{n}$ for all $n$, so the sequence is monotone and by Theorem 2.4.2, the
  sequence must converge and $\lim x_n$ exists. Now, $x_{n+1} = \sqrt{2 +
    x_{n}}$, So,
  \begin{align*}
    \lim x_{n+1}^2 &= \lim(2 + x_{n}) \\
    \lim x_{n}^2 &= 2 + \lim x_{n} \tag{By the ALT and since $\lim x_{n+1} = \lim x_n$}\\
    \lim x_n^2 - \lim x_n - 2 = 0 \\
    (\lim x_n +1)(\lim x_n -2) = 0
  \end{align*}
  Hence, $\lim x_n = 2$.
  \item The $n$-th term in the sequence contains $n$ (nested) square roots and
    assume $n > 3$. Then,
    \begin{align*}
      x_n &= \sqrt{2 \sqrt{2 \sqrt{2} \cdots }} = \prod_{k=1}^{n} 2^{1/{2^k}} = 2^{\sum_{k=1}^n 1/2^k}
    \end{align*}
    But $\sum_{k=1}^{\infty} 1/{2^k} = 1$, so that means that the sequence
    $(1/2^k)$ must be bounded, and, correspondingly that $x_n$ must be bounded.
    Hence, $\lim x_n$ must exist. Namely, $\lim x_n = 2$.
  \end{enumerate}
\end{ex}

\begin{ex}{2.4.4}

  \begin{enumerate}[label=\alph*)]
  \item Suppose that $\mathbb{N}$ is bounded from above and consider the
    sequence $(n)$. Clearly, $(n)$ is monotone so by the Monotone Congergence
    Theorem, $\lim n$ exists and we set $\lim n = \alpha$. Now, by the Algebraic
    Limit Theorem, $\lim (n + 1) = \lim n + 1 = \alpha + 1$. But $\lim n = \lim
    (n + 1)$, hence $\alpha = \alpha + 1$, which is a contradiction. We conclude
    that $\mathbb{N}$ is unbounded from above.
  \item In the proof of Theorem 1.4.1, we consider the sequence $(a_n)$, which
    is clearly monotone and bounded above by any $b_n$. Hence the limit $\lim
    a_n = \alpha$ must exist.

    To complete the proof,
    we need to show that for any convergent increasing sequence $(x_n)$, $x_n
    \leq \lim x_n$ for all $n$. Suppose there exists an $n_0$ such that $\lim
    x_n < x_{n_0}$. Since the sequence is increasing, this require that for all
    $n$, $\lim x_n < x_{n + n_0}$. But $\lim x_n = \lim x_{n + n_0}$ and, by
    the Order Limit Theorem, $\lim x_n < \lim x_{n + n_0} = \lim_x$, which is a
    contradiction.

    So, we now have that $a_n \leq \alpha \leq b_n$ for all $n$, as required,
    and the rest of the proof follows just as it did in the AoC version.
  \end{enumerate}
\end{ex}

\begin{ex}{2.4.5}
  \begin{enumerate}[label=\alph*)]
  \item Clearly, $x_1^2 = 4 \geq 2$. For the iterative case, we'll show that $x_{n+1}^2 -2 \geq 0$.
    \begin{align*}
      x_{n+1}^2 - 2 &= \left(\frac{1}{2}\left(x_n + \frac{2}{x_n}\right)\right)^2 - 2\\
                   &= \frac{1}{4}\left(x_n^2 + 4 + \frac{4}{x_n^2}\right) - 2\\
                   &= \frac{x_n^2}{4} - 1 + \frac{1}{x_n^2} \\
                   &= \left(\frac{x_n}{2} - \frac{1}{x_n} \right)^2 \\
                   &\geq 0
    \end{align*}
The fact that the sequence is decreasing (i.e., $x_n - x_{n+1} \geq 0$ for all $n \in \mathbb{N}$), now follows:
    \begin{align*}
      x_n - x_{n+1} &= x_n - \frac{1}{2}\left(x_n + \frac{2}{x_n}\right) \\
                   &= \frac{x_n^2 - 2}{2 x_{n}} \\
                   &\geq 0 \tag{since $x_{n}^2 \geq 2$}
    \end{align*}
  By the Monotone Convergence Theorem, the sequence converges and $x = \lim(x_n)$ for some $x$. By the definition of convergence
  and the Algebraic Limit Theorem,
\begin{align*}
 x &= \lim (x_n) \\
   &= \lim (x_{n+1}) \\
   &= \lim \left(\frac{1}{2} \left(x_n + \frac{2}{x_n} \right) \right) \\ 
   &= \frac{1}{2} \lim(x_n) + \frac{1}{\lim(x_n)} \\
   &= \frac{x}{2} + \frac{1}{x}
\end{align*}
this means that $0 = \frac{1}{x} - \frac{x}{2}$ or $x^2 = 2$, which yields that $x = \sqrt{2}$.
 \item The sequence
   \begin{align*}
   x_1 &= c^2  \\
   x_{n+1} &= \lim \left(\frac{1}{2} \left(x_n + \frac{c}{x_n} \right) \right)
   \end{align*}
converges to $\sqrt{c}$, which we can show by following the same steps from part a). Showing that $x_{n+1}^2 \geq c$ for all $n$:
    \begin{align*}
      x_{k+1}^2 - c &= \left(\frac{1}{2}\left(x_k + \frac{c}{x_k}\right)\right)^2 - c\\
                   &= \frac{1}{4}\left(x_k^2 + 2c + \frac{c^2}{x_k^2}\right) - c\\
                   &= \frac{1}{4}\left(x_k^2 - 2c + \frac{c^2}{x_k^2}\right) \\
                   &= \frac{1}{4}\left(x_k - \frac{c}{x_k}\right)^2 \\
                   &\geq 0
    \end{align*}
Showing the sequence is decreasing:
    \begin{align*}
      x_n - x_{n+1} &= x_n - \frac{1}{2}\left(x_n + \frac{c}{x_n}\right) \\
                   &= \frac{x_n}{2} - \frac{c}{2x_n} \\
                   &= \frac{x_n^2 - c}{2 x_{n}} \\
                   &\geq 0 \tag{since $x_{n}^2 \geq c$}
    \end{align*}
And finally, we have that

\begin{align*}
 x &= \lim (x_n) \\
   &= \lim (x_{n+1}) \\
   &= \lim \left(\frac{1}{2} \left(x_n + \frac{c}{x_n} \right) \right) \\ 
   &= \frac{1}{2} \lim(x_n) + \frac{c}{2\lim(x_n)} \\
   &= \frac{x}{2} + \frac{c}{2x}
\end{align*}
which means that $0 = \frac{c}{2x} - \frac{x}{2}$ or $x^2 = c$, which yields that $x = \sqrt{c}$.
 \end{enumerate}
\end{ex}

\begin{ex}{2.4.7}
  \begin{enumerate}[label=\alph*)]
    \item  Since $(a_n)$ is bounded there exists an $M > 0$ such that $|a_n| \leq M$ for all $n \in \mathbb{N}$. This means that each $\{a_k : k \geq n\}$ is bounded above by $M$. Now, 
\[
\{a_k : k \geq n\} \subseteq \{a_k : k \geq n +1\}
\]
so by Exercise 1.3.11 a (which states that for all nonempty, bounded sets $A, B$ with $A \subseteq B$, $\sup A \leq \sup B$), we have
$y_n \leq y_{n+1}$, so $(y_n)$ is decreasing. Additionally, $(y_n)$ is bounded by $M$ since each is $\{a_k : k \geq n\}$. By the Monotone Convergence Theorem, $(y_n)$ converges.

  \item The \emph{limit inferior} of $(a_n)$, or $\lim \inf a_n$, is defined by
\[
  \lim \inf a_n = \lim z_n,
\]
where $z_n = \inf\{a_k \mid k \geq n\}$. The argument that $z_n$ converges is virtually identical to the argument in a), except with $\sup$ replaced with $\inf$ and $\leq$ replaced with $\geq$.

\item  For any non-empty, bounded set $A$, $\inf A \leq \inf A$. (Suppose this were not the case: then there would exist a least lower bound $i$ and least upper bound $s$ such that $i \leq a$ for all $a \in A$ and $i > s$. But, $a \leq s$ so $i \leq a \leq s$, which is a contradiction.) Hence, $z_n \leq y_n$ for all $n$---by the Order Limit Theorem, $\lim \inf a_n \leq \lim \sup a_n$.

The inequality is strict for ``oscillating'' sequences, like $((-1)^n)$.

\item ($\Longrightarrow$) Suppose $\lim \inf a_n = \lim \sup a_n = a$. Let $\varepsilon > 0$ and choose $n \in N$ such that $|z_n - a| < \varepsilon$ and $|y_n -a| < \varepsilon$ (such a choice is possible by the convergence of the respective sequences). 

As a quick detour that I thought was relevant to this problem but actually isn't and should be ignored entirely, note that for any bounded and decreasing (and hence convergent) sequence $(b_n)$ with limit $b$, $b_n \geq b$ for all $n$. To prove this, suppose that there exists an $m$ such that $b_m < b$. Then for all $n \geq m$, we have $b_n < b$ since the sequence is decreasing. Since $(b_n)$ is convergent, there exists an $N \in \mathbb{N}$ such that for all $n \geq N$ we have $b - b_n \leq b - b_m$. But for any $n \geq m$ we have
$b_m \geq b_n$, and hence $b - b_m \leq b - b_n \leq b - b_m$ which requires that $b_n = b_m$. But then $b - b_n$ is a constant and thus must be 0, i.e. $b = b_n$, which is a contradiction. A similar claim for increasing sequences can be proved almost identically.

Since $z_n \leq x \leq y_n$ for all $x \in \{a_k \mid k \geq n \}$, $|x - a| < \varepsilon$ (since $z_n -a \leq x - a \leq y_n -a < \varepsilon$ and $\varepsilon > a - z_n \geq a -x \geq a - y_n$). Hence, $(a_n)$ converges to $a$.

($\Longleftarrow$) Suppose $\lim a_n = a$. Then, for any $\varepsilon > 0$, there exists $N$ such that $|a_n - a| < \varepsilon$ for all $n \geq N$. But then 
\begin{align*}
-\varepsilon &< a_n - a < \varepsilon \\
a -\varepsilon &< a_n < a + \varepsilon 
\end{align*}
which means that $a - \varepsilon \leq y_n \leq a + \varepsilon$ and $a - \varepsilon \leq z_n \leq a + \varepsilon$. But, then, by the Order Limit Theorem,
$a - \varepsilon \leq y \leq a + \varepsilon$ and $a - \varepsilon \leq z \leq a + \varepsilon$ and so, by Theorem 1.2.6, we conclude that  $\lim \inf a_n = \lim \sup a_n = a$.
  \end{enumerate}

\end{ex}

\begin{ex}{2.4.9}
  Suppose $\sum_{n=0}^{\infty} 2^nb_{2^n}$ diverges and consider the partial sums of the sequence $(s_m)$ (corresponding to the sum $\sum_{n=1}^{\infty} b_n$) indexed by $2^k$:

\begin{align*}
  s_{2^k} &= b_1 + b_2 + \left(b_3 + b_4\right) + \cdots + \left( b_{2^{k-1} +1} + \cdots + b_{2^k}\right) \\
         &\geq b_1 + b_2 + \left(b_4 + b_4\right) + \cdots + \left( b_{2^{k-1}} + \cdots + b_{2^k}\right)  \tag {since $(b_n)$ is decreasing} \\
         &= b_1 + b_2 + 2b_4 + \cdots + 2^{k-1}\left( b_{2^{k-1}} + \cdots + b_{2^k}\right)  \\
         &= \frac{1}{2} \left(2\left(b_1 + b_2 + 2b_4 + \cdots + 2^{k-1}\left( b_{2^{k-1}} + \cdots + b_{2^k}\right)\right)\right) \\
         &= \frac{1}{2} \left(b_1 + \left(b_1 + 2b_2 + 4b_4 + \cdots + 2^{k}\left( b_{2^{k-1}} + \cdots + b_{2^k}\right)\right)\right) \\
         &= \frac{1}{2} \left(b_1 + t_k\right)
\end{align*}
where $t_k$ is the $k$-th partial sum of $\sum_{n=0}^{\infty} 2^nb_{2^n}$. Now, since $t_k$ is unbounded, $s_{2^k}$ must be unbounded and we conclude that  $\sum_{n=1}^{\infty} b_n$ diverges.
\end{ex}

\begin{ex}{2.5.1}
  \begin{enumerate}[label=\alph*)]
  \item Impossible. By the Bolzano-Weierstrass Theorem, the bounded subsequence
    contains a convergent subsequence, which is also a subsequence of the original sequence.
  \item By Example 2.5.3, let $0 < b < 1$, then $(b^n)$ converges to 0. By the Algebraic Limit Theorem,
    $(1 - b^n)$ converges to 1. Hence, the sequence
    \[
       (b^0, 1 - b^0, b^1, 1 - b^1, \dots)
    \]
    contains subsequences which converge to both 0 and 1.
  \item The sequence
    \[
      (1, 1, 1/2, 1, 1/2, 1/3, 1, 1/2, 1/3, 1/4, \dots)
    \]
    works.
  \item Impossible, there will always be a subsequence that converges
    to 0. Consider the intervals $I_k = [0, 1/k]$. Each $I_k$ contains
    an infinite number of points. (Otherwise the sequence would not
    contain a subsequence which converges to $1/(k +1)$.)  Choose
    $a_{n_k} \in I_{k+1}$. Then, $|a_{n_k} - 0| < 1/k$.  Let
    $\varepsilon > 0$ and choose $N$ such that $N > 1/\varepsilon$. Then,
    for all $l > N$, we have $|a_{n_l} - 0| < 1/l < \varepsilon$. Hence,
    $(a_{n_k}) \rightarrow 0$.
  \end{enumerate}
\end{ex}

\begin{ex}{2.5.2}
  \begin{enumerate}[label=\alph*)]
  \item True. The subsequence $(x_2, x_3, \dots)$ is a proper subsequence and converes to the same
    limit as $(x_n)$.
  \item True. If $(x_n)$ converged to some $a$, then for any
    $\varepsilon > 0$, all but a finite number of terms would be
    contained in $V_\varepsilon(a)$, including all the terms of the
    divergent subsequence. But this implies that the divergent
    subsequence converges, which is a contradiction.
  \item True. Suppose that there exists only a single subsequence of
    $(x_n)$ which converges. This implies that in the interval halving
    process described in the proof of Theorem 2.5.5, all but a finite
    number of points are contained in a single half for every
    $I_k$--otherwise there would be an interval in which both halves
    contain an infinite number of points, in which case the process
    described would yield two convergent subsequences by application
    to each half.

    But this implies that the sequence converges to to the same value as the subsequence,
    which contradicts the fact that it's divergent.

  \item True. Suppose $(a_{n_k})$ converges to $a_s$. Then, for any
    $\varepsilon > 0$, there exists an $K$ such that for all $k > K$,
    $a_{n_k} \in V_{\varepsilon}(a_s)$. Let $m > n_k$. Then, (taking
    $(a_n)$ to be decreasing without loss of generality) there exists
    a $k'$ such that $a_{n_k} \leq a_m \leq a_{n_{k'}}$ since $(a_n)$ is decreasing.
    But this implies that $a_m \in V_\varepsilon(a_s)$.
  \end{enumerate}
\end{ex}

\begin{ex}{2.5.3}
  \begin{enumerate}[label=\alph*)]
  \item Let
    \begin{align*}
      b_1 &= a_1 + \cdots + a_{n_1} \\
      b_2 &= a_{n_1 + 1} + \cdots + a_{n_2} \\
          &\quad\vdots
    \end{align*}
    Clearly, the partial sums of $(b_n)$ is a subsequence of the partial sums of $(a_n)$. Since the
    series $(a_n)$ converges, by Theorem 2.5.2, so does the series $(b_n)$.
  \item The infinite series at the end of Section 2.1 doesn't converge.
  \end{enumerate}
\end{ex}

\begin{ex}{2.5.4}
  Let $S \subset \mathbb{R}$ be bounded from above by $M$ and let $a_1 \in S$. Consider $I_1 = [a_1 , M]$ and construct
  $I_2$ by bisecting $I_1$ and choosing the right half (i.e., $[(M-a_1)/2, M]$) if there exists an $a_2 \in S, [(M-a_1)/2, M]$. Otherwise, choose the left half. Continuing in the fashion, we obtain a series of nested intervals $I_k$, the intersection of which the Nested Interval Property guarantees is non-empty--let $x \in I_k$ for all $k$. Let $a \in S$. If $a \leq a_1$, clearly $x \geq a$. If $a \geq a_1$, then $a \in [a_1, M]$. Consider the largest $k$ for which
  $a \in I_k$. That is, $a \notin I_{l}$ with $l > k$. The only way this can happen is if $a$ is in the left bisection
  of $I_k$ and the right bisection is chosen. So, in this case, $x \geq a$. If $a \in I_k$ for all $k$, then $a = x$. This follows since the length of the interval $I_k$ is $(M-a_1)/2^n$ and by the Algebraic Limit Theorem, $\left((M-a_1)/2^n\right) \rightarrow 0$.

  Hence, for any $a \in S$, $x \geq a$ and we've established $x$ as an upper bound for $S$. Now consider an upper bound $y$ for $S$ and suppose $y < x$. This requires that there exists a $k$ such that $y$ is on the left bisection of $I_k$ and $x$ is on the right with $y$ being strictly less than the midpoint. (If this were not the case, then both $x$ and $y$ would be in $I_k$ for all $k$, and we'd have $x = y$.) But the right bisection is only chosen when there exists an $a_k \in S$ in the right bisection, meaning that $y < a_k$, which contradicts the fact that $y$ is an upper bound for $S$. We conclude that $x \leq y$, establishing $x$ as the supremum of $S$ as required.
\end{ex}

\begin{ex}{2.5.5}
  By the Bolzano-Weierstrass Theorem, $(a_n)$ contains a convergent subsequence. Now, consider the proof of the Bolzano-Weierstrass Theorem. At each interval $I_k$, we're tasked with bisecting it and choosing a half which contains an infinite number of terms. Suppose both halves of $I_k$ contain an infinite number of terms.

  If there are an infinite number of terms equal to the midpoint $m_k$ of $I_k$, then there exists a bounded sequence converging to $m_k$ and hence, by assumption, all subsequences converge to $m_k$. But this means that, excluding the midpoint, each half only contains a finite number of points--otherwise both the left and right bisections of $I_{k+1}$ will contain an infinite number of points and hence yield bounded sequences converging to different values, which is a contradiction. By a nearly identical argument, both halves of $I_k$ cannot contain an infinite number of terms. Hence, for any $k$, there exists an $N$ such that for all $n > N$, $a_n \in I_k$. Let $\varepsilon > 0$ and choose $N$ such that $a_n \in I_k$ for all $k > \log_2 (M/\varepsilon)$ and $n > N$. Then,
  \[
  |I_k| = M/2^k < M/2^{(\log_2(M/\varepsilon))} = \varepsilon
  \]
Hence, $|a_n - a| < \varepsilon$ and we have $(a_n) \rightarrow a$.
\end{ex}

\begin{ex}{2.6.1}
  Suppose $(x_n) \rightarrow x$. Let $\varepsilon > 0$. Since $(x_n)$ converges, there exists an $N$ such that
  for all $n,m > N$, $|x_n - x| < \varepsilon/2$ and $|x_m - x| < \varepsilon/$. Then,
  \begin{align*}
    |x_n - x_m| &= |x_n - x + x - x_m|  \\
    &\leq |x_n -x| + |x_m - x| \\
    &< \epsilon/2 + \epsilon/2 \\
    &= \epsilon.
  \end{align*}
  Hence, $(a_n)$ is a Cauchy sequence.
\end{ex}

\begin{ex}{2.6.2}
  \begin{enumerate}[label=\alph*)]
  \item $(1, -1, 1/2, -1/2, 1/4, -1/4, \dots)$ does the trick.
  \item Impossible. By Theorem 2.5.2, subsequence of convergent sequences are convergent and by
    the Cauchy Criterion, a Cauchy sequence is a convergent sequence.
  \item Impossible. A divergent monotone sequence is unbounded, but
    Lemma 2.6.3 says that Cauchy sequences are bounded. All
    subsequences $(a_{n_k})$ of divergent monotone sequences must also
    diverge, otherwise (assuming an increasing sequence without loss
    of generality) there would exist an $M$ with $a_{n_k} \leq M$ for
    all $k$, which implies that $a_n \leq M$ for all $n$, since for
    any $a_n$, there exists a $k$ such that $a_n \leq a_{n_k}$.
  \item $(1, 2, 1, 3, 1, 4, 1, 5, \dots)$ does the trick.
  \end{enumerate}
\end{ex}

\begin{ex}{2.6.7}
  \begin{enumerate}[label=\alph*)]
  \item Let $(a_n)$ be monotone and bounded by $M$. With loss of generality, assume $(a_n)$ is increasing. By the
    Bolzano-Weierstrass Theorem, there exists a convergent subsequence $(a_{n_k})$. By the argument in
    Exercise 2.6.2 c), $(a_n)$ converges.
  \item Let $(a_n)$ be bounded by $M$. Consider the interval $I_0 = [-M, M]$. Bisecting this interval,
    we obtain two halves $[-M, M/2]$ and $[M/2, M]$, at least one of which will contain an infinite number of points
    in the sequence, which we call $I_1$. $I_k$ is constructed by $k$ such bisections.

    Consider the subsequence $(a_{n_k})$ where $a_{n_k} \in I_k$. Let
    $\varepsilon > 0$ and choose $K \in \mathbb{N}$ such that $K >
    1 -\log_2 (M/\varepsilon)$, which we can always do by the
    Archimedian Property.  then, for all $k, j > K$, we have $M/2^{k-1}
    < \varepsilon$ and $M/2^{j-1} < \varepsilon$ and hence for any
    $a_{n_k}, a_{n_j} \in I_K$, we have $|a_{n_k} - a_{n_j}| < \varepsilon$ since
    $|I_K| = M/2^{K-1}$. Hence, $(a_{n_k})$ is Cauchy and by the Cauchy Criterion, converges.

  \item The Archimedian Property is true on $\mathbb{Q}$ but the Axiom of Completeness does not hold on
    $\mathbb{R}$, so clearly the Archimedian Property is insufficient to prove the Axiom of Completeness.
  \end{enumerate}
\end{ex}

\begin{ex}{2.7.1}
  \begin{enumerate}[label=\alph*)]
  \item Let $\varepsilon > 0$. Since $(a_n)$ is decreasing and $(a_n)
    \rightarrow 0$, it's easy to show that $a_n \geq 0$ for all $n$. There exists an $N$ such
    that for all $n \geq N$, $a_n < \epsilon$ since $(a_n) \rightarrow 0$. Choose $m \geq N$ such
    that $m$ is odd. Then, since $(a_n)$ is decreasing,
    \begin{align*}
    &a_m - a_{m+1} + a_{m+2} - a_{m+3} + a_{m+4} - \dots \pm a_{m+k} \\
      &\leq \begin{cases}
        a_m - a_{m+2} + a_{m+2} - a_{m+4} + a_{m+4} \dots + a_{m+k} & \text{if $k$ is even} \\
        a_m - a_{m+2} + a_{m+2} - a_{m+4} + a_{m+4} \dots - a_{m+k} & \text{if $k$ is odd} \\
      \end{cases}  \\
      &= \begin{cases}
        a_m & \text{if $k$ is even} \\
        a_m - a_{m+k} & \text{if $k$ is odd} \\
      \end{cases} \\
      &< \varepsilon
    \end{align*}
    Hence, $(s_n)$ is Cauchy.

  \item Define
    \begin{align*}
    I_0 &= [0, s_1] \\
    I_k &= \begin{cases}
      [s_k, s_{k-1}] & \text{$k$ even} \\
      [s_{k-1}, s_k] & \text{$k$ odd}
           \end{cases}
    \end{align*}
    Then
    \begin{align*}
    |I_k| &= |s_k - s_{k-1}| = a_k
    \end{align*}
    and hence $(|I_k|) \rightarrow 0$. By the Nested Interval Property, there exists
    an $s \in I_k$ for all $k$. Clearly $s_n \in I_K$ for $n \geq K$. For
    any $\varepsilon > 0$, you can choose a $k$ such that $|I_k| < \varepsilon$. Hence,
    $s_n \in V_{\varepsilon}(s)$ for all $n > k$.
  \item Note,
    \begin{align*}
      s_{2n} &= a_1 - a_2 + \dots + a_{2n - 1} - a_{2n} \\
      s_{2n + 1} &= s_{2n} + a_{2n+1}
    \end{align*}
    Now, it follows from part a) that both $s_{2n}$ and $s_{2n+1}$ are
    increasing. Furthermore, $s_{2n} \leq a_1$ and $s_{2n+1} \leq
    a_1$. So, both sequences are bounded and increasing and by the
    Monotone Convergence Theorem, must converge. Additionally, both
    sequences must converge to the same limit since $s_{2n} - s_{2n+1}
    = a_{2n+1}$, so $(s_{2n} - s_{2n+1}) \rightarrow 0$. But since
    $s_n = s_{2(n/2)}$ for even $n$ and $s_n = s_{2((n-1)/2) + 1}$ for
    odd $n$, $s_n$ must also converge to the same limit as the two
    subsequences.
  \end{enumerate}
\end{ex}

\begin{ex}{2.7.2}
  \begin{enumerate}[label=\alph*)]
  \item $1/(2^n + n) < 1/2^n$ for all $n \in \mathbb{N}$. $\sum_{n=1}^{\infty} 1/2^n$ converges by Example 2.7.5, so
    by the Comparison Test, $\sum_{n=1}^{\infty} 1/(2^n+n)$ converges.
  \item Since $-1 \leq \sin (n) \leq 1$, this series is bounded above by $\sum_{n=1}^{\infty} 1/n^2$, which we know converges. Hence, the series converges.
  \item Presumably diverges since $(a_n) \rightarrow 1/2$, but it's unlcear to me how to actually show this
    since the Alternating Series Test says nothing about necessisity.
  \end{enumerate}
\end{ex}

\begin{ex}{2.7.9}
  \begin{enumerate}[label=\alph*)]
  \item Suppose there exists no $N$ such that $n \geq N$ implies $|a_{n+1}| \leq |a_n| r'$.
    \[
       \lim \left|\frac{a_{n+1}}{a_n} \right| = r
       \]
       means that for any $\varepsilon > 0$, there exists an $N$ such that $n \geq N$ implies
       \[
       \left| \left|\frac{a_{n+1}}{a_n} \right| - r \right| < \varepsilon
       \]
       so
       \[
       - (\varepsilon + r) < \left|\frac{a_{n+1}}{a_n} \right| < \varepsilon + r
       \]
       which we relax to obtain
       \[
       \left|\frac{a_{n+1}}{a_n} \right| \leq \varepsilon + r
       \]
       Since $0 \leq r < r' < 1$, we can choose $\varepsilon = r' - r$ and obtain
       \[
       \left|\frac{a_{n+1}}{a_n} \right| \leq r'
       \]
       as required.
     \item $0 < r' < 1$, so this converges by Example 2.7.5.
     \item From part a), there exists an $N$ such that for all $n \geq N$, we have
       $|a_{n+1}| \leq |a_n|r'$. But then
       \[
          |a_{N+m}| \leq |a_{N+(m-1)}|r' \leq |a_{N+(m-2)}|(r')^2 \leq \dots \leq |a_N|(r')^m
          \]
       Hence, by the Comparison Test and part b) and the Absolute Convergence Test, the series converges.
  \end{enumerate}
\end{ex}

\begin{ex}{2.7.13}
  \begin{enumerate}[label=\alph*)]
  \item \begin{align*}
    \sum_{k=1}^{n} x_ky_k &= s_ny_{n+1} - s_{0}y_0 + \sum_{k=1}^{n}s_k(y_k - y_{k+1}) \\
                       &= s_ny_{n+1} + \sum_{k=1}^{n}s_k(y_k - y_{k+1}) \\
  \end{align*}
  \item Since $(s_n) \rightarrow s$, there exists a $B \geq 0$ such that $s_n \leq B$ for all $n$. (Since convergent sequences
    are bounded.) Additionally, $(y_k)$ is decreasing. Hence,
    \begin{align*}
      s_k (y_k - y_{k+1}) \leq
      B (y_k - y_{k+1}) \leq
      B (y_1 - y_{k+1}) 
    \end{align*}
    The series $\sum B (y_1 - y_{k+1})$ converges by the Algebraic Limit Theorem and hence
    $\sum s_k (y_k - y_{k+1})$ converges absolutely, since $y_k - y_{k+1} \geq 0$, by the Comparison Test.

    This immediately leads to a proof of Abel's Test by the Algebraic Limit Theorem.
  \end{enumerate}
\end{ex}

\begin{ex}{2.8.1}
  We have
  \begin{align*}
    s_{nn} = \sum_{j=1}^{n} \left(\frac{-1}{2^{j-1}}\right) = \sum_{k=0}^{n} (-1) (\frac{1}{2})^k
  \end{align*}
  which is a geometri series and hence converges to -2, which matches the iterated limit
  that summed the columns first.
\end{ex}

\begin{ex}{2.8.2}
  By the absolute convergence test, for any $i \in \mathbb{N}$, $\sum_{j=1}^{\infty} |a_{ij}|$ converges to
  some number $c_i$. Now, $\sum_{j=1}^{\infty} a_{ij} \leq \sum_{j=1}^{\infty} |a_{ij}|$ so $c_i \leq b_i$ for all $i$.
  By the Comparison Test, $\sum_{i=1}^{\infty} c_i$ converges and we conclude that the iterated series
  $\sum_{i=1}^{\infty} \sum_{j=1}^{\infty} a_{ij}$ converges.
\end{ex}

\begin{ex}{2.8.3}
  \begin{enumerate}[label=\alph*)]
  \item Since $\sum_{i=1}^{\infty} \sum_{j=1}^{\infty} |a_{ij}|$
    converges, for each $i \in \mathbb{N}$, $\sum_{j=1}^{\infty}
    |a_{ij}|$ converges to some $b_i$ and $\sum_{i=1}^{\infty} b_i$ converges to some $b$.

    Now, since $|a_{i,j}| \geq 0$ we have
    \[
      \sum_{j=1}^n |a_{ij}| \leq b_i
      \]
      for each $i$. Hence,
      \[
      \sum_{i=1}^n \sum_{j=1}^n |a_{i,j}| \leq 
      \sum_{i=1}^n \sum_{j=1}^n b_i \leq
      b
      \]
      and we conclude that $(t_{nn})$ is bounded by $b$. But $(t_{nn})$ is increasing, and hence it must
      converge.
    \item
      Since $(t_{nn})$ is Cauchy, for any $\varepsilon > 0$, there exists an $N$ such that for all $m, n \geq N$,
      \begin{align*}
        \left|t_{nn} - t_{mm}\right| &< \varepsilon \\
      \end{align*}
      Without loss of generality, suppose $n \geq m$. Then, 
      \begin{align*}
        \left|s_{nn} - s_{mm}\right|
        &= \left|\sum_{i=1}^n \sum_{j=1}^n a_{i,j} - \sum_{i=1}^m \sum_{j=1}^m a_{i,j}\right| \\
        &= \left| \sum_{i=m+1}^n \sum_{j=1}^n a_{i,j} + \sum_{i=1}^m \sum_{j=m+1}^n a_{i,j} \right| \\
        &\leq \left| \sum_{i=m+1}^n \sum_{j=1}^n |a_{i,j}| + \sum_{i=1}^m \sum_{j=m+1}^n |a_{i,j}| \right| \\
        &= \left| t_{nn} - t_{mm} \right| \\
        &< \varepsilon
      \end{align*}
      showing that $(s_{nn})$ is Cauchy as well.
  \end{enumerate}
\end{ex}

\begin{ex}{2.8.4}
  \begin{enumerate}[label=\alph*)]
  \item Let $\varepsilon > 0$. Suppose there does not exist an $N_1$
    such that $m, n \geq N_1$ implies that $B - \varepsilon/2 <
    t_{mn}$. Then, for any choice of $N_1$, there exists $m, n \geq
    N_1$ such that $B - \varepsilon/2 \geq t_{mn}$. Let $m_1, n_1$ be
    arbitrary and consider $t_{m_1n_1}$. There exists $m_2 \geq m_1$
    and $n_2 \geq n_1$ such that $B -\varepsilon/2 \geq t_{m_2n_2}$.
    Furthermore, since $(t_{mn})$ is increasing, $t_{m_2n_2} \geq t_{m_1n_1}$.
    Hence,
    \[
       B-\varepsilon/2 \geq t_{m_1n_1}
       \]
       But this means that $B -\varepsilon/2$ is an upper bound of $(t_{mn})$, which
       is a contradiction.
     \item We have
       \begin{align*}
         | s_{mn} - S | &= | s_{mn} - s_{nn} + s_{nn} - S| \\
         &\leq | s_{mn} - s_{nn} | + | s_{nn} - S| \\
         &\leq \left| \sum_{m=n+1}^{\infty} \sum_{n=1}^{\infty} a_{ik} \right| + |s_{nn} - S| \\ \tag{assuming $m \geq n$}
         &\leq \left| \sum_{m=n+1}^{\infty} \sum_{n=1}^{\infty} |a_{ik}| \right| + |s_{nn} - S| \\
         &\leq \left| t_{mn} - t_{nn} \right| + |s_{nn} - S| \\
       \end{align*}

       Now, from part a), we have for all $m \geq n \geq N_1$. So,
       \[
         B - \frac{\varepsilon}{2} < t_{nn} \leq t_{mn} \leq B
         \]
         hence
       \begin{align*}
         B - \frac{\varepsilon}{2} - t_{nn} &< 0 \leq t_{mn} - t_{nn} \leq B - t_{nn} \\
         B - \frac{\varepsilon}{2} - B < \leq t_{mn} - t_{nn} \leq B - B + \frac{\varepsilon}{2} \\
         \frac{\varepsilon}{2}  < t_{mn} - t_{nn} \leq  \frac{\varepsilon}{2} \\
       \end{align*}
       Now, since $(s_{nn}) \rightarrow S$, there exists $n \geq N_2$ such that $|s_{nn} - S| < \varepsilon/2$.
       Set $N = \max\{N_1, N_2\}$ and let $m \geq n > N$. Then,
       \begin{align*}
         | s_{mn} - S | &\leq \left| t_{mn} - t_{nn} \right| + |s_{nn} - S| \\
         &< \frac{\varepsilon}{2} + \frac{\varepsilon}{2} \\
         &< \varepsilon
       \end{align*}
  \end{enumerate}
\end{ex}
\end{document}


